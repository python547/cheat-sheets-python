\documentclass[10pt,landscape,a4paper]{article}
\usepackage[right=10mm, left=10mm, top=10mm, bottom=10mm]{geometry}
\usepackage[utf8]{inputenc}
\usepackage[T1]{fontenc}
\usepackage[english]{babel}
\usepackage[rm,light]{roboto}
\usepackage{xcolor}
\usepackage{graphicx}
\graphicspath{{./figures/}}
\usepackage{multicol}
\usepackage{colortbl}
\usepackage{array}
\setlength\parindent{0pt}
\setlength{\tabcolsep}{2pt}
\baselineskip=0pt
\setlength\columnsep{1em}
\definecolor{Gray}{gray}{0.85}

% --- Listing -----------------------------------------------------------------
\usepackage{listings}
\lstset{
  frame=tb, framesep=4pt, framerule=0pt,
  backgroundcolor=\color{black!5},
  basicstyle=\ttfamily,
  commentstyle=\ttfamily\color{black!50},
  breakatwhitespace=false,
  breaklines=true,
  extendedchars=true,
  keepspaces=true,
  language=Python,
  rulecolor=\color{black},
  showspaces=false,
  showstringspaces=false,
  showtabs=false,
  tabsize=2,
  %
  emph = { plot, scatter, imshow, bar, contourf, pie, subplots, show, savefig,
           errorbar, boxplot, hist, set_title, set_xlabel, set_ylabel, suptitle,  },
  emphstyle = {\ttfamily\bfseries}
}

% --- Fonts -------------------------------------------------------------------
\usepackage{fontspec}
\usepackage[babel=true]{microtype}
\defaultfontfeatures{Ligatures = TeX, Mapping = tex-text}
\setsansfont{Roboto} [ Path           = fonts/roboto/Roboto-,
                       Extension      = .ttf,
                       UprightFont    = Light,
                       ItalicFont     = LightItalic,
                       BoldFont       = Regular,
                       BoldItalicFont = Italic ]
\setromanfont{RobotoSlab} [ Path           = fonts/roboto-slab/RobotoSlab-,
                            Extension      = .ttf,
                            UprightFont    = Light,
                            BoldFont       = Bold ]
\setmonofont{RobotoMono} [ Path           = fonts/roboto-mono/RobotoMono-,
                           Extension      = .ttf,
                           Scale          = 0.90,
                           UprightFont    = Light,
                           ItalicFont     = LightItalic,
                           BoldFont       = Regular,
                           BoldItalicFont = Italic ]
\renewcommand{\familydefault}{\sfdefault}

% -----------------------------------------------------------------------------
\begin{document}
\thispagestyle{empty}

\section*{\LARGE \rmfamily
          Matplotlib \textcolor{orange}{\mdseries for beginners}}

\begin{multicols*}{3}

Matplotlib is a library for making 2D plots in Python. It is designed
with the philosophy that you should be able to create simple plots
with just a few commands:\\

\fbox{1} \textbf{Initialize}
\begin{lstlisting}
 import numpy as np
 import matplotlib.pyplot as plt
\end{lstlisting}
%
\fbox{2} \textbf{Prepare}
\begin{lstlisting}
 X = np.linspace(0, 4*np.pi, 1000)
 Y = np.sin(X)
\end{lstlisting}
%
\fbox{3} \textbf{Render}
\begin{lstlisting}
 fig, ax = plt.subplots()
 ax.plot(X, Y)
 fig.show()
\end{lstlisting}
%
\fbox{4} \textbf{Observe} \medskip\\
\includegraphics[width=\linewidth]{sine.pdf}

% -----------------------------------------------------------------------------
\subsection*{\rmfamily Choose}
% -----------------------------------------------------------------------------

Matplotlib offers several kind of plots (see Gallery): \medskip

\begin{tabular}{@{}m{.821\linewidth}m{.169\linewidth}}
\begin{lstlisting}[belowskip=-\baselineskip]
 X = np.random.uniform(0, 1, 100)
 Y = np.random.uniform(0, 1, 100)
 ax.scatter(X, Y)
\end{lstlisting}
& \raisebox{-0.75em}{\includegraphics[width=\linewidth]{basic-scatter.pdf}}
\end{tabular}
% -----------------------------------------------------------------------------
\begin{tabular}{@{}m{.821\linewidth}m{.169\linewidth}}
\begin{lstlisting}[belowskip=-\baselineskip]
 X = np.arange(10)
 Y = np.random.uniform(1, 10, 10)
 ax.bar(X, Y)
\end{lstlisting}
& \raisebox{-0.75em}{\includegraphics[width=\linewidth]{basic-bar.pdf}}
\end{tabular}
% -----------------------------------------------------------------------------
\begin{tabular}{@{}m{.821\linewidth}m{.169\linewidth}}
\begin{lstlisting}[belowskip=-\baselineskip]
 Z = np.random.uniform(0, 1, (8,8))

 ax.imshow(Z)
\end{lstlisting}
& \raisebox{-0.75em}{\includegraphics[width=\linewidth]{basic-imshow.pdf}}
\end{tabular}
% -----------------------------------------------------------------------------
\begin{tabular}{@{}m{.821\linewidth}m{.169\linewidth}}
\begin{lstlisting}[belowskip=-\baselineskip]
 Z = np.random.uniform(0, 1, (8,8))

 ax.contourf(Z)
\end{lstlisting}
& \raisebox{-0.75em}{\includegraphics[width=\linewidth]{basic-contour.pdf}}
\end{tabular}
% -----------------------------------------------------------------------------
\begin{tabular}{@{}m{.821\linewidth}m{.169\linewidth}}
\begin{lstlisting}[belowskip=-\baselineskip]
 Z = np.random.uniform(0, 1, 4)

 ax.pie(Z)
\end{lstlisting}
& \raisebox{-0.75em}{\includegraphics[width=\linewidth]{basic-pie.pdf}}
\end{tabular}
% -----------------------------------------------------------------------------
\begin{tabular}{@{}m{.821\linewidth}m{.169\linewidth}}
\begin{lstlisting}[belowskip=-\baselineskip]
 Z = np.random.normal(0, 1, 100)

 ax.hist(Z)
\end{lstlisting}
& \raisebox{-0.75em}{\includegraphics[width=\linewidth]{advanced-hist.pdf}}
\end{tabular}
% -----------------------------------------------------------------------------
\begin{tabular}{@{}m{.821\linewidth}m{.169\linewidth}}
\begin{lstlisting}[belowskip=-\baselineskip]
 X = np.arange(5)
 Y = np.random.uniform(0, 1, 5)
 ax.errorbar(X, Y, Y/4)
\end{lstlisting}
& \raisebox{-0.75em}{\includegraphics[width=\linewidth]{advanced-errorbar.pdf}}
\end{tabular}
% -----------------------------------------------------------------------------
\begin{tabular}{@{}m{.821\linewidth}m{.169\linewidth}}
\begin{lstlisting}[belowskip=-\baselineskip]
 Z = np.random.normal(0, 1, (100,3))

 ax.boxplot(Z)
\end{lstlisting}
& \raisebox{-0.75em}{\includegraphics[width=\linewidth]{advanced-boxplot.pdf}}
\end{tabular}


% -----------------------------------------------------------------------------
\subsection*{\rmfamily Tweak}
% -----------------------------------------------------------------------------
You can modify pretty much anything in a plot, including limits,
colors, markers, line width and styles, ticks and ticks labels,
titles, etc. \medskip

% -----------------------------------------------------------------------------
\begin{tabular}{@{}m{.821\linewidth}m{.169\linewidth}}
\begin{lstlisting}[belowskip=-\baselineskip]
 X = np.linspace(0, 10, 100)
 Y = np.sin(X)
 ax.plot(X, Y, color="black")
\end{lstlisting}
& \raisebox{-0.75em}{\includegraphics[width=\linewidth]{plot-color.pdf}}
\end{tabular}
% -----------------------------------------------------------------------------
\begin{tabular}{@{}m{.821\linewidth}m{.169\linewidth}}
\begin{lstlisting}[belowskip=-\baselineskip]
 X = np.linspace(0, 10, 100)
 Y = np.sin(X)
 ax.plot(X, Y, linestyle="--")
\end{lstlisting}
& \raisebox{-0.75em}{\includegraphics[width=\linewidth]{plot-linestyle.pdf}}
\end{tabular}
% -----------------------------------------------------------------------------
\begin{tabular}{@{}m{.821\linewidth}m{.169\linewidth}}
\begin{lstlisting}[belowskip=-\baselineskip]
 X = np.linspace(0, 10, 100)
 Y = np.sin(X)
 ax.plot(X, Y, linewidth=5)
\end{lstlisting}
& \raisebox{-0.75em}{\includegraphics[width=\linewidth]{plot-linewidth.pdf}}
\end{tabular}
% -----------------------------------------------------------------------------
\begin{tabular}{@{}m{.821\linewidth}m{.169\linewidth}}
\begin{lstlisting}[belowskip=-\baselineskip]
 X = np.linspace(0, 10, 100)
 Y = np.sin(X)
 ax.plot(X, Y, marker="o")
\end{lstlisting}
& \raisebox{-0.75em}{\includegraphics[width=\linewidth]{plot-marker.pdf}}
\end{tabular}


% -----------------------------------------------------------------------------
\subsection*{\rmfamily Organize}
% -----------------------------------------------------------------------------

You can plot several data on the the same figure, but you can also
split a figure in several subplots (named {\em Axes}): \medskip

% -----------------------------------------------------------------------------
\begin{tabular}{@{}m{.821\linewidth}m{.169\linewidth}}
\begin{lstlisting}[belowskip=-\baselineskip]
 X = np.linspace(0, 10, 100)
 Y1, Y2 = np.sin(X), np.cos(X)
 ax.plot(X, Y1, Y2)
\end{lstlisting}
& \raisebox{-0.75em}{\includegraphics[width=\linewidth]{plot-multi.pdf}}
\end{tabular}
% -----------------------------------------------------------------------------
\begin{tabular}{@{}m{.821\linewidth}m{.169\linewidth}}
\begin{lstlisting}[belowskip=-\baselineskip]
 fig, (ax1, ax2) = plt.subplots((2,1))
 ax1.plot(X, Y1, color="C1")
 ax2.plot(X, Y2, color="C0")
\end{lstlisting}
& \raisebox{-0.75em}{\includegraphics[width=\linewidth]{plot-vsplit.pdf}}
\end{tabular}
% -----------------------------------------------------------------------------
\begin{tabular}{@{}m{.821\linewidth}m{.169\linewidth}}
\begin{lstlisting}[belowskip=-\baselineskip]
 fig, (ax1, ax2) = plt.subplots((1,2))
 ax1.plot(Y1, X, color="C1")
 ax2.plot(Y2, X, color="C0")
\end{lstlisting}
& \raisebox{-0.75em}{\includegraphics[width=\linewidth]{plot-hsplit.pdf}}
\end{tabular}

% -----------------------------------------------------------------------------
\subsection*{\rmfamily Label \mdseries (everything)}
% -----------------------------------------------------------------------------
% -----------------------------------------------------------------------------
\begin{tabular}{@{}m{.821\linewidth}m{.169\linewidth}}
\begin{lstlisting}[belowskip=-\baselineskip]
 ax.plot(X, Y)
 fig.suptitle(None)
 ax.set_title("A Sine wave")
\end{lstlisting}
& \raisebox{-0.75em}{\includegraphics[width=\linewidth]{plot-title.pdf}}
\end{tabular}
% -----------------------------------------------------------------------------
\begin{tabular}{@{}m{.821\linewidth}m{.169\linewidth}}
\begin{lstlisting}[belowskip=-\baselineskip]
 ax.plot(X, Y)
 ax.set_ylabel(None)
 ax.set_xlabel("Time")
\end{lstlisting}
& \raisebox{-0.75em}{\includegraphics[width=\linewidth]{plot-xlabel.pdf}}
\end{tabular}

% -----------------------------------------------------------------------------
\subsection*{\rmfamily Explore}
% -----------------------------------------------------------------------------

Figures are shown with a graphical user interface that allows to zoom
and pan the figure, to navigate between the different views and to
show the value under the mouse.

% -----------------------------------------------------------------------------
\subsection*{\rmfamily Save \mdseries (bitmap or vector format)}
% -----------------------------------------------------------------------------
\begin{lstlisting}[belowskip=-\baselineskip]
 fig.savefig("my-first-figure.png", dpi=300)
 fig.savefig("my-first-figure.pdf")
\end{lstlisting}
%
\vfill
%
{\scriptsize Matplotlib 3.2 handout for beginners. Copyright (c)
  2020 Nicolas P. Rougier. Released under a CC-BY International 4.0
  License. Supported by NumFocus Grant \#12345.\par}

\end{multicols*}
\end{document}
