\documentclass[10pt,landscape,a4paper]{article}
\usepackage[right=10mm, left=10mm, top=10mm, bottom=10mm]{geometry}
\usepackage[utf8]{inputenc}
\usepackage[T1]{fontenc}
\usepackage[english]{babel}
\usepackage[rm,light]{roboto}
\usepackage{xcolor}
\usepackage{graphicx}
\graphicspath{{./figures/}}
\usepackage{multicol}
\usepackage{colortbl}
\usepackage{array}
\setlength\parindent{0pt}
\setlength{\tabcolsep}{2pt}
\baselineskip=0pt
\setlength\columnsep{1em}
\definecolor{Gray}{gray}{0.85}

% --- Listing -----------------------------------------------------------------
\usepackage{listings}
\lstset{
  frame=tb, framesep=4pt, framerule=0pt,
  backgroundcolor=\color{black!5},
  basicstyle=\ttfamily,
  commentstyle=\ttfamily\color{black!50},
  breakatwhitespace=false,
  breaklines=true,
  extendedchars=true,
  keepspaces=true,
  language=Python,
  rulecolor=\color{black},
  showspaces=false,
  showstringspaces=false,
  showtabs=false,
  tabsize=2,
  %
  emph = { plot, scatter, imshow, bar, contourf, pie, subplots, spines,
    add_gridspec, add_subplot, set_xscale, set_minor_locator,
    annotate, set_minor_formatter, tick_params, fill_betweenx, text, legend,
    errorbar, boxplot, hist, title, xlabel, ylabel, suptitle },
  emphstyle = {\ttfamily\bfseries}
}

% --- Fonts -------------------------------------------------------------------
\usepackage{fontspec}
\usepackage[babel=true]{microtype}
\defaultfontfeatures{Ligatures = TeX, Mapping = tex-text}
\setsansfont{Roboto} [ Path           = fonts/roboto/Roboto-,
                       Extension      = .ttf,
                       UprightFont    = Light,
                       ItalicFont     = LightItalic,
                       BoldFont       = Regular,
                       BoldItalicFont = Italic ]
\setromanfont{RobotoSlab} [ Path           = fonts/roboto-slab/RobotoSlab-,
                            Extension      = .ttf,
                            UprightFont    = Light,
                            BoldFont       = Bold ]
\setmonofont{RobotoMono} [ Path           = fonts/roboto-mono/RobotoMono-,
                           Extension      = .ttf,
                           Scale          = 0.90,
                           UprightFont    = Light,
                           ItalicFont     = LightItalic,
                           BoldFont       = Regular,
                           BoldItalicFont = Italic ]
\renewcommand{\familydefault}{\sfdefault}

% -----------------------------------------------------------------------------
\begin{document}
\thispagestyle{empty}

\section*{\LARGE \rmfamily
          Matplotlib \textcolor{orange}{\mdseries for intermediate users}}

\begin{multicols*}{3}

A matplotlib figure is composed of a hierarchy of elements that forms
the actual figure. Each element can be modified. \medskip

\includegraphics[width=\linewidth]{anatomy.pdf}

\subsection*{\rmfamily Figure, axes \& spines}

% -----------------------------------------------------------------------------
\begin{tabular}{@{}m{.821\linewidth}m{.169\linewidth}}
\begin{lstlisting}[belowskip=-\baselineskip]
 fig, axs = plt.subplots((3,3))
 axs[0,0].set_facecolor("#ddddff")
 axs[2,2].set_facecolor("#ffffdd")
\end{lstlisting}
& \raisebox{-0.75em}{\includegraphics[width=\linewidth]{layout-subplot-color.pdf}}
\end{tabular}

% -----------------------------------------------------------------------------
\begin{tabular}{@{}m{.821\linewidth}m{.169\linewidth}}
\begin{lstlisting}[belowskip=-\baselineskip]
 gs = fig.add_gridspec(3, 3)
 ax = fig.add_subplot(gs[0, :])
 ax.set_facecolor("#ddddff")
\end{lstlisting}
& \raisebox{-0.75em}{\includegraphics[width=\linewidth]{layout-gridspec-color.pdf}}
\end{tabular}

% -----------------------------------------------------------------------------
\begin{tabular}{@{}m{.821\linewidth}m{.169\linewidth}}
\begin{lstlisting}[belowskip=-\baselineskip]
 fig, ax = plt.subplots()
 ax.spines["top"].set_color("None")
 ax.spines["right"].set_color("None")
\end{lstlisting}
& \raisebox{-0.75em}{\includegraphics[width=\linewidth]{layout-spines.pdf}}
\end{tabular}



% -----------------------------------------------------------------------------
\subsection*{\rmfamily Ticks \& labels}

\begin{lstlisting}[basicstyle=\ttfamily\small]
 from mpl.ticker import MultipleLocator as ML
 from mpl.ticker import ScalarFormatter as SF
 ax.xaxis.set_minor_locator(ML(0.2))
 ax.xaxis.set_minor_formatter(SF())
 ax.tick_params(axis='x',which='minor',rotation=90)
\end{lstlisting}
\includegraphics[width=\linewidth]{tick-multiple-locator.pdf}

% -----------------------------------------------------------------------------
\subsection*{\rmfamily Lines \& markers}

\begin{lstlisting}
 X = np.linspace(0.1, 10*np.pi, 1000)
 Y = np.sin(X)
 ax.plot(X, Y, "C1o:", markevery=25, mec="1.0")
\end{lstlisting}
\includegraphics[width=\linewidth]{sine-marker.pdf}

% -----------------------------------------------------------------------------
\subsection*{\rmfamily Scales \& projections}

\begin{lstlisting}
 fig, ax = plt.subplots()
 ax.set_xscale("log")
 ax.plot(X, Y, "C1o-", markevery=25, mec="1.0")
\end{lstlisting}
\includegraphics[width=\linewidth]{sine-logscale.pdf}

\subsection*{\rmfamily Text \& ornaments}
\begin{lstlisting}[]
 ax.fill_betweenx([-1,1],[0],[2*np.pi])
 ax.text(0, -1, r" Period $\Phi$")
\end{lstlisting}
\includegraphics[width=\linewidth]{sine-period.pdf}


% -----------------------------------------------------------------------------
\subsection*{\rmfamily Legend}
\begin{lstlisting}[]
 ax.plot(X, np.sin(X), "C0", label="Sine")
 ax.plot(X, np.cos(X), "C1", label="Cosine")
 ax.legend(bbox_to_anchor=(0,1,1,.1),ncol=2,
           mode="expand",  loc="lower left")
\end{lstlisting}
\includegraphics[width=\linewidth]{sine-legend.pdf}

% -----------------------------------------------------------------------------
\subsection*{\rmfamily Annotation}
\begin{lstlisting}[]
 ax.annotate("A", (X[250],Y[250]),(X[250],-1),
   ha="center", va="center",arrowprops =
   {"arrowstyle" : "->", "color": "C1"})
\end{lstlisting}
\includegraphics[width=\linewidth]{sine-annotate.pdf}

% -----------------------------------------------------------------------------
\subsection*{\rmfamily Colors}

Any color can be used, but Matplotlib offers sets of colors:\\
\includegraphics[width=\linewidth]{colors-cycle.pdf} \smallskip
\includegraphics[width=\linewidth]{colors-grey.pdf}\\
%As well as nice colormaps (viridis an magma):\\
%\includegraphics[width=\linewidth]{colormap-viridis.pdf} \smallskip
%\includegraphics[width=\linewidth]{colormap-magma.pdf} \medskip

% -----------------------------------------------------------------------------
\vspace{-1em}
\subsection*{\rmfamily Size \& DPI}

Consider a square figure to be included in a two-columns A4 paper with
2cm margins on each side and a column separation of 1cm. The width of
a figure is (21 - 2*2 - 1)/2 = 8cm. One inch being 2.54cm, figure size
should be 3.15$\times$3.15 in.
\begin{lstlisting}[]
 fig = plt.figure(figsize=(3.15,3.15), dpi=50)
 plt.savefig("figure.pdf", dpi=600)
\end{lstlisting}


\vfill
%
{\scriptsize
  Matplotlib 3.5.0 handout for intermediate users.
  Copyright (c) 2021 Matplotlib Development Team.
  Released under a CC-BY 4.0 International License.
  Supported by NumFOCUS.
\par}



\end{multicols*}
\end{document}
